\usefonttheme{professionalfonts}
\usetheme{Antibes}
\setbeamertemplate{footline}[frame number]

\usepackage[
    backend=biber,
    style=alphabetic,
    sorting=ynt
]{biblatex}
\addbibresource{../bib/bibliografia.bib}

\makeatletter
\setbeamertemplate{footline}
{
    \leavevmode%
    \hbox{%
        \begin{beamercolorbox}[wd=.333333\paperwidth,ht=2.25ex,dp=1ex,center]{author in head/foot}%
            \usebeamerfont{author in head/foot}\insertshortauthor
        \end{beamercolorbox}%
        \begin{beamercolorbox}[wd=.333333\paperwidth,ht=2.25ex,dp=1ex,center]{title in head/foot}%
            \usebeamerfont{title in head/foot}\insertshorttitle
        \end{beamercolorbox}%
        \begin{beamercolorbox}[wd=.333333\paperwidth,ht=2.25ex,dp=1ex,right]{date in head/foot}%
            \usebeamerfont{date in head/foot}\insertshortdate{}\hspace*{2em}
            \insertframenumber{} / \inserttotalframenumber\hspace*{2ex} 
        \end{beamercolorbox}}%
        \vskip0pt%
    }
    \makeatother

%% Itemize bullet head

% set the itemize item symbol as a triangle
\setbeamertemplate{itemize item}{\scriptsize$\blacktriangleright$}
% set the itemize subitem symbol as a diamond
\setbeamertemplate{itemize subitem}{\scriptsize$\diamond$}
% set the itemize subsubitem symbol as a circle with a dot
\setbeamertemplate{itemize subsubitem}{\scriptsize$\odot$}

\usepackage[portuguese]{babel}
\usepackage{csquotes}
\usepackage{graphicx}
\usepackage{amsmath, mathtools, mathrsfs, amssymb}
\usepackage{verbatim}
\usepackage{pgfplots}
\usepackage{xcolor}
    \pgfplotsset{compat=1.18}
    \pgfplotsset{lua debug}
\usepackage{tikz}
\usepackage{graphicx}
\usetikzlibrary{positioning, matrix}
\usepackage{bm}
\usepackage{hyperref}
\hypersetup{
    colorlinks=true,
    citecolor=blue,
    linkcolor=blue,
    filecolor=magenta,
    urlcolor=blue,
    pdfpagemode=FullScreen,
}

%% Supress overful hbox warnings until this lenght
\hfuzz=3.22pt 
\vfuzz=4.6pt 

\DeclareMathOperator{\proj}{proj \ }
\renewcommand{\arraystretch}{1.4}
\newcommand{\transpose}{\mathsf{T}}

%% Environments

\theoremstyle{plain} % default
\newtheorem{teo}{Teorema}[section]
\newtheorem*{teo*}{Teorema}
\newtheorem{lem}{Lema}[section]
\newtheorem*{lem*}{Lema}
\newtheorem{prop}{Proposição}[section]
\newtheorem{cor}[teo]{Corolário}
\newtheorem*{cor*}{Corolário}
\newtheorem*{axiom}{Axioma}

\newtheorem*{TAU}{Teorema da Aproximação Universal}
\newtheorem*{Riesz}{Teorema da Representação de Riesz}

\theoremstyle{definition}
\newtheorem{defn}{Definição}[section]
\newtheorem*{defn*}{Definição}
\newtheorem{conj}{Conjectura}[section]
\newtheorem{exmp}{Exemplo}[section]
\newtheorem{rem}{Observação}[section]
\newtheorem*{rem*}{Observação}

\theoremstyle{remark}
% \newtheorem*{note}{Nota}
\newtheorem{case}{Caso}


% Macros

\renewcommand{\Re}{\text{Re}}

\newcommand{\bfx}{\bm{x}}
\newcommand{\bfX}{\bm{X}}
\newcommand{\bfy}{\bm{y}}
\newcommand{\bfz}{\bm{z}}
\newcommand{\bfw}{\bm{w}}
\newcommand{\bfv}{\bm{v}}

\newcommand{\bvx}{\vec{\bm{x}}}


\newcommand{\bH}{\mathbb{H}}
\newcommand{\K}{\mathbb{K}}
\newcommand{\I}{\mathbb{I}}

\DeclarePairedDelimiter{\dotprod}{\langle}{\rangle}

\DeclareMathOperator{\rk}{rk}
\DeclareMathOperator{\intt}{int}
\DeclareMathOperator{\diam}{diam}
\DeclareMathOperator{\rref}{rref}
\DeclareMathOperator{\vspan}{span}
\DeclareMathOperator{\lin}{Lin}
\DeclareMathOperator{\supp}{supp}



\DeclareMathOperator{\posto}{posto}

% Swap the definition of \abs* and \norm*, so that \abs
% and \norm resizes the size of the brackets, and the 
% starred version does not.
